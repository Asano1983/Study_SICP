
シンクホーンアルゴリズムにより得た近似解を主問題の解に変換する。

シンクホーンアルゴリズムにより、双対問題の解と目的関数を得ることができるが、
主問題の解(輸送行列)を得たい場合もしばしばある。

無限回の反復が可能ならば、双対解は厳密解となり、式(3.18)を用いて主問題の解を得ることができるが、現実的には有限回で反復を打ち止めることになる。
本節では厳密解ではない双対解を式(3.18)により変換した後に、輸送行列にうまく丸める方法を紹介する。
疑似コードをアルゴリズム3.4に示す。

TODO:アルゴリズム3.4(任意の正行列を輸送行列に丸める)

このアルゴリズムの正当性は以下の定理により示される。

TODO:定理3.5

ゆえに実行可能とは限らない主問題の解$A$があったとき、
$A$の目的関数値が小さく、かつ$A$の行和・列和が$a$,$b$に十分近ければ、
アルゴリズム3.4によって実行可能かつ目的関数値が小さい解を計算できる。
例えば、シンクホーンアルゴリズムの反復途中の$u^{(k)}$,$v^{(k)}$を用いて、
式(3.18)により$A=...$と変換した行列を$A$として用いることができる。

以降で解析するように、シンクホーンアルゴリズムにより得られる解は目的関数値が最適値に近く(定理3.9)、
$A$の行和・列和が$a$,$b$に近い(定理3.11)ことが保証されるので、
丸めの結果も良い輸送行列となります。

シンクホーンアルゴリズムの出力を丸める手法は次節における計算量の議論でも使用する。



シンクホーンアルゴリズムの大域収束性

エントロピー正則化つきの最適輸送問題の双対問題の目的関数を
指数領域での変数$u=\exp(f/\varepsilon)$とv=\exp(g/\varepsilon)$について書き表すと

\[ L_D(u,v) = a^T (\varepsilon \log u) + b^T (\varepsilon \log v) - \varepsilon \sum_{ij} u_i K_{ij} v_j \]
となる。

2.3.5節で述べたように、通常の最適輸送問題では双対問題の変数に定数成分の自由度がある。
これはエントロピー正則化つき問題についても同様。
すなわち$u \in R^n$,$v \in R^m$を双対問題の解とすると、定数$c \in R$について$u' = cu$、$v' = \frac{1}{c} v$は等価な解とみなせる。

このような定数成分の違いを吸収できる正ベクトルの距離として、以下で定義されるヒルベルト距離がある。







